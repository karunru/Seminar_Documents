\section{Factorization Machines}
\subsection{Model Equation}

式(\ref{eq:fm})に,FMの次元$d = 2$のときのモデルの式を示す.
\begin{equation}
  \hat{y}(\vtr{x}) = w_{0} + \sum_{i=1}^{n}w_{i}{x}_{i} + \sum_{i=1}^{n} \sum_{j=i+1}^{n} \langle \vtr{v}_{i}, \vtr{v}_{j} \rangle{x}_{i}{x}_{j} \label{eq:fm}
\end{equation}
また,推定すべきFMのパラメーターは以下である.
\begin{equation}
    w_{0} \in \R, \vtr{w} \in \R^{n}, \vtr{V} \in \R^{n \times k}
\end{equation}
ここで, $\langle \cdot, \cdot \rangle$ は$k$次元の2つのベクトルの内積を表す.
\begin{equation}
  \langle \vtr{v}_{i}, \vtr{v}_{j} \rangle = \sum_{f=1}^{k} v_{i,f} \cdot v_{j,f}
\end{equation}

$\vtr{v}_{i}$は$V$の$i$行目の$k$次元のベクトルである.
$k \in \mathbb{N}^{+}$は分解の次元を定義するハイパーパラメーターである.

次元$d=2$のFMは2つの変数間の関係を捉えることができ,
$w_{0}$は全体のバイアス,$w_{i}$は$i$番目の変数の強さ,
$\hat{w}_{i, j} := \dt<\vtr{v}_{i}, \vtr{v}_{j}>$は変数間の関係を表す.
$w_{i, j} \in \R$を使わず各変数に設定された重みの内積で表現することにより,
学習データが少ないスパースなデータでも効率よく学習することができる.


\subsection{計算量}

式(\ref{eq:fm})の計算量は,$\mathcal{O}(kn^{2})$であるが,式(\ref{eq:fm_order})による式変換を行うことにより,$\mathcal{O}(kn)$に削減することができる.
\begin{eqnarray}
  && \sum_{i=1}^{n} \sum_{j=i+1}^{n} \langle \vtr{v}_{i}, \vtr{v}_{j} \rangle{x}_{i}{x}_{j} \nonumber \\
  &=& \frac{1}{2} \sum_{i=1}^{n} \sum_{j=1}^{n} \langle \vtr{v}_{i}, \vtr{v}_{j} \rangle{x}_{i}{x}_{j} - \frac{1}{2} \langle \vtr{v}_{i}, \vtr{v}_{i} \rangle{x}_{i}{x}_{i} \nonumber \\
  &=& \frac{1}{2} \left( \sum_{i=1}^{n} \sum_{j=1}^{n} \sum_{f=1}^{k} v_{i,f} v_{j,f}{x}_{i}{x}_{j} - \sum_{i=1}^{n} \sum_{f=1}^{k} v_{i,f} v_{i,f}{x}_{i}{x}_{i} \right) \nonumber \\
  &=& \frac{1}{2} \sum_{f=1}^{k} \left ( \left ( \sum_{i=1}^{n} v_{i,f}{x}_{i} \right ) \left ( \sum_{j=1}^{n} v_{j,f}{x}_{j} \right ) - \sum_{i=1}^{n} v_{i,f}^{2}{x}_{i}^{2} \right) \nonumber \\
  &=& \frac{1}{2} \sum_{f=1}^{k} \left ( \left ( \sum_{i=1}^{n} v_{i,f}{x}_{i} \right )^{2}  - \sum_{i=1}^{n} v_{i,f}^{2}{x}_{i}^{2} \right ) \label{eq:fm_order}
\end{eqnarray}

また,入力ベクトル$\vtr{x}$がスパースであるとすると,非ゼロ成分の数だけ計算をすればいいので,
計算量は$\mathcal{O}(k\overline{m}_{D})$まで削減できる.

\subsection{d-way FM}
$d$次元のFMの式を式(\ref{eq:d-wayFM})に示す.
\begin{equation}
  \hat{y}(\vtr{x}) = w_{0} +
  \sum_{i=1}^{n} w_{i} x_{i} +
  \sum_{l=2}^{d} \sum_{i_{1}=1}^{n} \cdots \sum_{i_{l}=i_{l-1}+1}^{n}
  \left ( \prod_{j=1}^{l} x_{i_{j}}  \right )
  \left ( \sum_{f=1}^{k_{l}} \prod_{j=1}^{l} v_{i_{j}, f}^{(l)} \right ) \label{eq:d-wayFM}
\end{equation}

また,パラメータは以下である.
\begin{equation}
    w_{0} \in \R, \vtr{w} \in \R^{n}, \forall l \in \{2, \cdots, d \} : \vtr{V}^{(l)} \in \R^{n \times k_{l}}, k_{l} \in \mathbb{N}^{+}
\end{equation}

$d$次元のFMも,単純に計算すれば$\mathcal{O}(k_{d}n^{d})$だが,式(\ref{eq:fm_order})と同様の式変形を行うことにより,
線形時間で計算することが可能である.

\subsection{パラメータの学習}
FMのパラメータは,SGD(Stochastic Gradient Descent)をはじめとした勾配法,
ALS(Alternating least-squares)\footnote{交互最小二乗法.NMFのパラメータ求めるときみたいにあるパラメータを更新するときはそれ以外のパラメータを固定してってやるやつ},
MCMC(Markov Chain Monte Carlo)による推定の3種類がRendleによるライブラリに実装されている\cite{Rendle:2012:FML:2168752.2168771}.
ここでは,正則化項なしの単純なSGDによるパラメータ推定を紹介する.

まず,問題設定として以下の最適化問題を定義する.
\begin{equation}
  \argmin_{\Theta} \sum_{(\vtr{x}, y) \in D} \ell(\hat{y}(\vtr{x} \mid \Theta), y)
\end{equation}
$\Theta$はモデルのパラメータをまとめたものであり,FMの次元$d=2$のとき$\Theta = \{ w_{0}, \vtr{w}, V \}$である.
また,$\hat{y}(\vtr{x} \mid \Theta)$はパラメータ$\Theta$のもとで入力$\vtr{x}$のときの予測値を表す.

この最適化問題では,誤差関数$\ell$を選択することができる.
回帰を行う際には式(\ref{eq:LeastSquareLoss})に示す最小二乗誤差関数(Least Square Loss)を,
分類を行う際には式(\ref{eq:LogisticLoss})に示すロジスティック誤差関数(Logistic Loss)またはhinge lossを用いる.
\begin{eqnarray}
  \ell_{\rm LS}(t, \hat{y}) &=& (\hat{y} - t)^{2},\hspace{8ex} t \in \R \label{eq:LeastSquareLoss} \\
  \ell_{\rm C}(t, \hat{y}) &=& -\ln(\sigma(t\cdot \hat{y})) \nonumber \\
  &=& -\ln \left ( \frac{1}{1 + \exp(-t\hat{y})} \right ) \nonumber \\
  &=& \ln(1 + \exp(-t\hat{y})),\hspace{8ex}  t \in \{-1, 1\} \label{eq:LogisticLoss}
\end{eqnarray}
ここで,$t$は教師データ,$\hat{y} \in \R$はモデルの予測値を表す.
また,$\sigma(x) = \displaystyle \frac{1}{1 + \exp(-x)}$はシグモイド関数である.

誤差関数を直接最適化するために,それぞれの導関数を求める.
\begin{equation}
  \frac{\partial}{\partial \theta} \ell_{\rm LS}(\hat{y}(\vtr{x} \mid \Theta), y)
  = \frac{\partial}{\partial \theta}(\hat{y}(\vtr{x} \mid \Theta) - y)^{2}
  = 2(\hat{y}(\vtr{x} \mid \Theta) - y)\frac{\partial}{\partial \theta}\hat{y}(\vtr{x} \mid \Theta)
\end{equation}
\begin{eqnarray}
  \frac{\partial}{\partial \theta} \ell_{\rm C} (\hat{y}(\vtr{x} \mid \Theta), y) &=& \frac{\partial}{\partial \theta} -\ln(\sigma(\hat{y}(\vtr{x} \mid \Theta) \cdot y)) \nonumber \\
  &=& - \frac{1}{\sigma(\hat{y}(\vtr{x} \mid \Theta) \cdot y)}\sigma(\hat{y}(\vtr{x} \mid \Theta) \cdot y)(1-\sigma(\hat{y}(\vtr{x} \mid \Theta) \cdot y)\frac{\partial}{\partial \theta}\hat{y}(\vtr{x} \mid \Theta) \nonumber \\
  &=& (\sigma(\hat{y}(\vtr{x} \mid \Theta) \cdot y) - 1)\frac{\partial}{\partial \theta}\hat{y}(\vtr{x} \mid \Theta)
\end{eqnarray}
また,FMの各パラメータにおける勾配を式(\ref{eq:fm_gradient_1})〜式(\ref{eq:fm_gradient_3})に示す.
\begin{eqnarray}
  \frac{\partial}{\partial w_{0}} \hat{y}(\vtr{x}) &=& 1 \label{eq:fm_gradient_1}\\
  \frac{\partial}{\partial w_{i}} \hat{y}(\vtr{x}) &=& x_{i} \\
  \frac{\partial}{\partial v_{i,f}} \hat{y}(\vtr{x}) &=& \frac{\partial}{\partial v_{i,f}} \frac{1}{2} \sum_{f'=1}^{k} \left ( \left ( \sum_{j=1}^{n} v_{j,f'}{x}_{j} \right )^{2}  - \sum_{j=1}^{n} v_{j,f'}^{2}{x}_{j}^{2} \right ) \nonumber \\
  &=& \frac{\partial}{\partial v_{i,f}} \frac{1}{2} \left ( \left ( \sum_{j=1}^{n} v_{j,f}{x}_{j} \right )^{2} - v_{i,f}^{2}{x}_{i}^{2} \right ) \nonumber \\
  &=& \frac{1}{2} \left ( 2 \left ( \sum_{j=1}^{n} v_{j,f}{x}_{j} \right ) \cdot x_{i} - 2 v_{i,f}{x}_{i}^{2} \right ) \nonumber \\
  &=& x_{i} \displaystyle \sum_{j=1}^{n} v_{j,f} x_{j} - v_{i,f} x_{i}^{2} \label{eq:fm_gradient_3}
\end{eqnarray}
式(\ref{eq:fm_gradient_3})に含まれる$\sum_{j=1}^{n} v_{j,f} x_{j}$は,$\hat{y}(\vtr{x})$を計算する際に予め計算しておくことが可能であるため,
全てのパラメータの勾配は$\mathcal{O}(1)$で求めることができる.

SGDでは,以下の更新式に従ってパラメータを更新していく.
\begin{equation}
  \theta \gets \theta - \eta \left ( \frac{\partial}{\partial \theta} \ell(\hat{y}(\vtr{x}), y) \right )
\end{equation}
ここで,データ$(\vtr{x}, y) \in D$は,ランダムに選択される(または1周する毎にデータ$D$をシャッフルする).
また,$\eta \in \R^{+}$は学習率を決めるパラメータである.

アルゴリズム\ref{alg1}に,正則化項なしのSGDによるFMのパラメータ推定を示す.

\begin{algorithm}[H]
\caption{Stochastic Gradient Descent(SGD) of Factorization Machines(FM)}
\label{alg1}
  \begin{algorithmic}
  \Require{Training data $D$, Learning rate $\eta$, Initialization $\sigma$}
  \Ensure{Model parameters $\Theta = \{ w_{0}, \vtr{w}, V \}$}
  \State{$w_{0} \gets 0; \vtr{w} \gets (0, \cdots, 0); V \sim \mathcal{N}(0, \sigma);$}
  \Repeat
    \For{$(\vtr{x}, y) \in D$}
      \State{$w_{0} \gets w_{0} - \eta \left ( \frac{\partial}{\partial w_{0}} \ell(\hat{y}(\vtr{x} \mid \Theta), y)  \right )$}
      \For{$i \in \{ 1, \cdots, n \} \land x_{i} \ne 0$}
        \State{$w_{i} \gets w_{i} - \eta \left ( \frac{\partial}{\partial w_{i}} \ell(\hat{y}(\vtr{x} \mid \Theta), y)  \right )$}
        \For{$f \in \{ 1, \cdots, k \}$}
          \State{$v_{i, f} \gets v_{i, f} - \eta \left ( \frac{\partial}{\partial v_{i, f}} \ell(\hat{y}(\vtr{x} \mid \Theta), y)  \right )$}
        \EndFor
      \EndFor
    \EndFor
  \Until{stopping criterion is met;}
  \end{algorithmic}
\end{algorithm}
