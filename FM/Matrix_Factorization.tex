\section{Matrix Factorization}
\begin{figure}[t]
  \centering \includegraphics[width=1\hsize, bb=0 0 736 426]{fig/MF.pdf}
  \caption{Diagram of Matrix Factorization.}
  \label{fig:MF}
\end{figure}

Matrix Factorization(MF)は,$m \times n$の行列を$m \times k$と$k \times n$の2つの行列に分解し,
その積で元の式を近似することで,元の行列では未知(0)であった要素を推測する手法である\cite{Koren:2009:MFT:1608565.1608614}.
図\ref{fig:MF}に,MFの概要図を示す.

$m \times n$の行列を$R$,$m \times k$の行列を$Q$,$k \times n$の行列を$P$とすると,
\begin{equation}
  R \approx QP
\end{equation}
と表せる.

$QP=\hat{R}$とすると,$\hat{R}$の要素$\hat{r}_{ij}$は
\begin{equation}
  \hat{r}_{ij} = \vtr{q}_{i}^{\top} \vtr{p}_{j} = \sum_{f=1}^{k} q_{if} p_{fj}
\end{equation}
と表される.ここで,$\vtr{q}_{i}$は$Q$の$i$行目のベクトル,$\vtr{p}_{j}$は$P$の$j$列目のベクトルである.

$R$を近似する2つの行列$Q$と$P$を求めるには,$R$と$\hat{R}$の誤差が最小になるような$Q$と$P$を求めればよい.

\subsection{Non-negative Matrix Factorization}
Non-negative Matrix Factorization(NMF)は,MFの元の行列の全ての要素$r_{ij}$が0以上であるとき,
分解される行列$Q$と$P$の各要素も全て0以上であるという制約を設けたMFである\cite{NIPS2000_1861}.

NMFでは,$R$と$\hat{R}$の誤差をユークリッド距離で測るとき,以下の式で$Q$と$P$を求める.

\begin{eqnarray}
  P_{ij}^{(n+1)} &=& P_{ij}^{(n)} \frac{\left[ Q^{(n)\top} R \right]_{ij}}{\left[ Q^{(n)\top} Q^{(n)} P^{(n)} \right]_{ij}} \\
  Q_{ij}^{(n+1)} &=& Q_{ij}^{(n)} \frac{\left[ R P^{(n+1)\top} \right]_{ij}}{\left[ Q^{(n)} P^{(n+1)} P^{(n+1)\top} \right]_{ij}}
\end{eqnarray}
ここで,$\left[ \cdot \right]_{ij}$は,$\left[ \  \right]$内の計算によって得られる行列の$i,j$要素を表す.

これらの更新式で,値が収束するまで更新することで,$Q$と$P$が求まる.

\subsection{Weighted Non-negative Matrix Factorization}
Weighted Non-negative Matrix Factorization(WNMF)は,欠損値が含まれている行列に対して
NMFを行うための手法である\cite{wnmf}.
WNMFでは,NMFに対しマスク行列$W$を追加する.
マスク行列$W$の各要素$W_{ij}$は次のように与えられる.
\begin{equation}
  W_{ij} =
  \begin{cases}
    1 & \text{if $R_{ij}$ is observed} \\
    0 & \text{if $R_{ij}$ is unobserved}
  \end{cases}
\end{equation}

マスク行列$W$を用いたWNMFの更新式は以下で与えられる.
\begin{eqnarray}
  P_{ij}^{(n+1)} &=& P_{ij}^{(n)} \frac{\left[ Q^{(n)\top} (W \circ R) \right]_{ij}}{\left[ Q^{(n)\top} (W \circ (Q^{(n)} P^{(n)})) \right]_{ij}} \nonumber \\
  && \\
  Q_{ij}^{(n+1)} &=& Q_{ij}^{(n)} \frac{\left[ (W \circ R) P^{(n+1)\top} \right]_{ij}}{\left[ (W \circ (Q^{(n)} P^{(n+1)})) P^{(n+1)\top} \right]_{ij}} \nonumber \\
  &&
\end{eqnarray}
ここで,$\circ$はアダマール積を表す.
